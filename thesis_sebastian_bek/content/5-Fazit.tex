\chapter{Fazit}
\label{chap:faz}
Das Ziel dieser Arbeit war es, ein algorithmisches Verfahren zur Analyse von Gruppierungs- und Bewegungsmustern zu entwickeln. Dieses sollte zunächst Trajektorien von Personen erfassen und analysieren. Aus den Trajektorien sollten weiterhin Messgrößen abgeleitet und dargestellt werden, mit denen eine möglichst genaue Analyse von Bewegungs- und Gruppierungsmustern möglich ist. Das entworfene Verfahren basiert auf dem Harris Corner Detector und dem sog. optischen Fluss (vgl. Kapitel \ref{chap:grund}). Es wurde im Rahmen eines Videoanalysesystems entwickelt, dessen Struktur in Kapitel \ref{chap:imp} näher beschrieben wurde. Im Kapitel \ref{chap:erg} wurde das entwickelte Verfahren mithilfe von Grundwahrheiten quantitativ bewertet. Dabei wurde festgestellt, dass die generierten relativen Messwerte mit dem Verhalten der manuell generierten Wahrheit (Ground Truth) korrelieren. Die generierten Messwerte können also als aussagekräftig betrachtet werden. 

Probleme die zeitweilige Abweichungen verursachen sind \zb Verdeckungen, die auftreten wenn im Verlauf der Sequenz die Personendichte stark variiert. Dieses Problem ist auf dem Gebiet der Bildverarbeitung bekannt und es existieren teilweise sehr aufwändige aber wirksame Lösungsansätze. Außerdem tritt oft zu Beginn der Sequenz ein Einschwingen bestimmter Messgrößen auf. Dies ist aber für die vorliegende Aufgabenstellung unkritisch, da das System für ein Langzeitmonitoring ausgelegt ist. Das vorliegende System wurde primär für die Erfassung bewegter Personen entwickelt und es konnte gezeigt werden, dass die Erfassung von Trajektorien in Personengruppen hoher Dichte, auch unter Verwendung von niedrig aufgelösten Bilddaten, zuverlässig funktioniert.

Als Ausblick könnte das Verfahren durch die Nutzung von intrinsischen und extrinsischen Kameraparametern, welche in der Regel durch eine Kalibrierung während der Einrichtphase gewonnen werden, hinsichtlich Systemparametrisierung und Robustifizierung verbessert werden. So könnten Evaluations-/Filtereinstellungen und Grenzwerte über den Bildort $x,y$ skaliert werden, um Einflüsse der unterschiedlich hohen Skalierung von Personen einzuschränken. Außerdem könnte versucht werden, dass Tracking der Personen durch die Wahl eines anderen Corner Detectors zu verbessern. 
\newpage
Möglicherweise existieren andere Eckendetektoren (\zb SIFT, SURF, ORB uvm.), die in diesem Kontext eine höhere Erkennungsleistung liefern. Um stehenbleibende Personen mitzuzählen, ohne zugleich statische Objekte mit zu erfassen, könnte die maximale Verweilzeit nur auf Trajektorien angewandt werden, die bereits eine große Strecke zurückgelegt haben. So hätten spontan an Objektkanten entstandene Trajektorien keine Berechtigung weiter zu existieren. Zusätzlich könnte das Verfahren mit einem vereinfachten Konfigurationsablauf sowie einer Konfigurationsdokumentation implementiert werden.

Abschließend kann festgestellt werden, dass das entwickelte Verfahren vielversprechende Ergebnisse liefert und in einer weiteren Entwicklungsstufe unter realen Bedingungen evaluiert werden sollte. Insbesondere sollten Crowd Manager(Sicherheitspersonal bei Großveranstaltungen) die Praxistauglichkeit des Verfahrens bewerten und dadurch die Einsetzbarkeit in der Praxis bestätigen.

%So könnte sie verbreitet genutzt werden, ohne dass ein zu großer Einrichtungsaufwand besteht. Solche Überwachungssysteme müssten, am Veranstaltungsort einmal installiert, vor jeder Veranstaltung einmal kalibriert werden und könnten im Verlauf der Veranstaltung für eine erhöhte Sicherheit sorgen. 
%Zusätzlich könnten die Veranstaltungsplaner im Nachhinein besonders kritische Gebiete des Veranstaltungsgeländes markieren, die entlastet werden sollten. Das sogenannte Tracking von Personen/Objekten kann auch in anderen Forschungsgebieten eingesetzt werden. Beispielsweise kann das Verhalten von Konsumenten in Supermärkten analysiert werden, um die Anordnung der Produkte zu optimieren oder auf kleinerer Skala könnten Proben im Labor unter dem Mikroskop hinsichtlich ihres Bewegungs- und Gruppierungsverhaltens analysiert werden. Diese und viele weitere Möglichkeiten bietet die algorithmische Erfassung von Trajektorien. Zum Einsatz einer solchen Software muss aber noch viel Forschung betrieben und gegebenenfalls müssen umfangreiche Praxistests durchgeführt werden. Die Arbeit hat jedoch gezeigt, dass die Erfassung präziser Trajektorien, nach ausgewähltem Verfahren, in Gebieten hoher Personendichte, auch mit niedrig aufgelösten Daten, grundsätzlich möglich ist. Außerdem wurde bestätigt, dass relevante, aussagekräftige Messgrößen aus den erstellten Trajektorien abgeleitet und dargestellt werden können. Demnach ist das Verfahren zur verbesserten Analyse von Gruppierungs- und Bewegungsmustern geeignet, weswegen eine weitere Behandlung des Themas sinnvoll ist.
%

%mnr: Du bist hier etwas ausgeschweift, und hast über mögliche Anwendungsgebiete gesprochen. Das sollte nicht sein, denn dann hättest du zunächst zeigen müssen, dass es für diese Anwendungen nicht bereits bessere Verfahren gibt. Wir bleiben also mit dieser Arbeit beim Thema Großveranstaltung!
%mnr: Hier wäre es besser wenn du noch Überlegungen anstellen könntest, wie man vielleicht einzelne Komponenten deines Verfahrens verbessern könnte. z.B. das Tracking, die Problematik "Person läuft irgendwo hin, und dann bleibt sie lange stehen"...
 
