\chapter*{Abstract}
\label{chap:Abstract}
%Sebastian
In den letzten Jahren erreichen uns immer häufiger Nachrichten über große Veranstaltungen, bei denen sich aufgrund zu hoher Besucherzahlen in Kombination mit unzureichender Planung Tumulte und Paniken, häufig mit Verletzten, ausbilden.
Solche Großveranstaltungen werden immer beliebter und sind fester Bestandteil der menschlichen Freizeitaktivitäten geworden, weswegen Einschränkungen Dieser unwahrscheinlich sind. Die Tendenz zeigt, das mit steigenden Besucherzahlen auch die Gefahr für Paniken und Unruhen steigt, weswegen videogestütztes Monitoring hier sinnvoll sein kann. Derzeit erfolgt das Monitoring von Veranstaltungen noch durch menschliche Betrachter, die oft unaufmerksam sind und meist nur subjektiv beurteilen können, wie hoch die Personenströme und -dichten wirklich sind. Zukünftig wird sich aber das automatisierte Monitoring durchsetzen, mit dem es möglich ist, ohne menschliches Eingreifen, Besucherzahlen und Personenströme quantitativ und relativ genau zu messen. Die automatische Erfassung und Analyse von Menschenströmen stellt jedoch zum heutigen Zeitpunkt noch eine technische Herausforderung dar.

In den ersten Jahren seines Lebens lernt der Mensch Formen, Strukturen und Kontouren richtig zu klassifizieren und aufgrund von Erfahrungswerten richtig einzuordnen. Mit diesen Erfahrungswerten, die sich über eine lange Entwicklung hin gebildet haben, kann der Mensch, mit einer sehr hohen Trefferquote, Personen als solche von anderen Strukturen in Bildern unterscheiden. Rechner und entsprechende Algorithmen sind hinsichtlich ihrer Erkennungsleistung noch lange nicht mit dem Menschen vergleichbar. Insbesondere bei niedriger Auflösung und starken Verdeckungsartefakten in der Szene, ist eine automatische Detektion von Personen mit den bekannten Verfahren bisher kaum möglich.

Im Gegenzug besitzt der Rechner, im Falle einer möglichen Detektion von Objekten, signifikante Vorteile bei der Analyse großer Datenmengen und bei der quantitativen Abschätzung von Messgrößen gegenüber der visuellen Auswertung. Um die Erkennungsleistung eines automatischen Systems, trotz der schwierigen Rahmenbedigungen bei der Beobachtung von Menschenmassen (niedrige Auflösung und Verdeckungen) auf ein einsetzbares Niveau zu steigern, werden im Rahmen dieser Arbeit Methoden basierend auf "`indirekten Merkmalen'', wie beispielsweise Bewegungsmerkmalen, untersucht, um auf Basis Dieser Informationen über die lokale Dichte und den lokalen Personenfluss innerhalb der Menschenmenge zu erschließen. Diese Dichte- und Flussinformationen sollen klassifiziert, quantifiziert und veranschaulicht werden, um etwaige Folgemaßnahmen bei kritischen Situationen zu erschließen. 
\newpage
Kurzfristige Folgemaßnahmen könnten \zb Evakuierung und Absperrung bestimmter Gebiete oder Eingriffe in Unruhen sein. Langfristige Folgemaßnahmen können \zb eine verbesserte Planung der Infrastruktur der Veranstaltung oder die Wahl eines größeren Veranstaltungsgeländes sein. Es soll ein Verfahren zur Erkennung und Messung der Bewegungs- und Gruppierungsmuster entwickelt und hinsichtlich seiner Performanz bewertet werden, um zu zeigen, dass es, auch unter Verwendung von niedrig aufgelösten Videodaten, für die Analyse von Dichte- und Flussinformationen in dicht gedrängten Personengruppen geeignet ist.
