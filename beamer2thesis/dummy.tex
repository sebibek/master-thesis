\documentclass[aspectratio=169,urlcolor=black]{beamer}
\usetheme[language=ngerman,
titlepagelogo=logopolito,
bullet=circle,
pageofpages=of,
color=blue,
titleline=true
]{TorinoTh}

%\usepackage[ngerman]{babel}
%\usepackage[utf8]{inputenc}
\usepackage{tabularx}
\usepackage{booktabs}
\usepackage{multicol}
\usepackage{ulem}
\usepackage{makecell}
\usepackage{upgreek}
\usepackage{movie15}
%\usepackage{xcolor}
%\usepackage{isodate}
\hypersetup{colorlinks,linkcolor=black,urlcolor=black}
\addto{\captionsngerman}{%
  \renewcommand*{\contentsname}{Contents}
  \renewcommand*{\listfigurename}{Figures}
  \renewcommand*{\listtablename}{Tables}
  \renewcommand*{\figurename}{Fig.}	
  \renewcommand*{\tablename}{Tab.}
}
\newcommand*\mean[1]{\bar{#1}}
\newcommand{\tabitem}{~~\llap{\textbullet}~~}
\newcommand\widebar[1]{\mathop{\overline{#1}}}

\usepackage{color}
\usepackage{graphicx}
\usepackage{fancybox}
\usepackage[singlespacing]{setspace}

\usepackage{beamerthemesplit}
\usetheme[compress]{Heidelberg}
\definecolor{unirot}{rgb}{0.4,0.4,0.3} % babyblue 0,0.58,1
\usecolortheme[named=unirot]{structure}
%\setbeamercolor{alerted text}{fg=red}
\newcommand*\hilite[1]{\textcolor{red}{#1}}
%\def\hilite<#1>{%
  %\temporal<#1>{\color{black}}{\color{unirot}}%
               %{\color{gray}}}


\title[Light Transport Techniques for Tensor Field Visualization]{Light Transport Techniques for Tensor Field Visualization}
\subtitle{Master's Thesis Presentation}
\vspace*{2pt}
\author[Sebastian Bek]{Sebastian Bek\vspace*{7pt}}
\date{July 24th 2019}
\institute[Uni HD]{
Heidelberg University\\
Visual Computing Group (VCG)\\
Supervisors: Prof. Filip Sadlo, Dr. Susanne Krömker\\
}

%\color{unirot}{sebibek@gmail.com}
%\newsubfloat{figure}
\newcommand{\source}[1]{\hspace{-3pt} {\tiny \raisebox{-0.75in}{\rotatebox[origin=t]{90}{Source: \color{black}{#1}}}}}
\setlength{\belowcaptionskip}{-10pt}
\setlength{\intextsep}{-20pt}

\begin{document}

%\frame[plain]{\titlepage}


\frame{
\frametitle{{Propagation Scheme - Physical Model}}
\begin{block}{\centering \textbf{Crystal Fiber Structures}}
\begin{itemize}
	\item crystal lattices reveal gradients dependent on direction, which leads to anisotropic light transport inside the medium (birefringence)
	\item the propability for redistribution of a photon in a particular direction is given by the phase function model yielded from Rayleigh scattering:
	\begin{align*}
 	P(\omega) &= \frac{T(\omega)}{\int_{k\pi}^{(k+1)\pi}T(\omega)\mathop{d\omega}}\\
 	\mathit{with} \int_0^{\pi} P(\omega) &= 1.
	\end{align*}
	\item each tensor is then a unique footprint: $t_{i,j}\mapsto T(\omega)$
\end{itemize}
\end{block}
} % END OF FRAME



%glyphs are used to represent the eigensystem, i.e., the anisotropy characteristics and/or tensor magnitude
%Volume non-directionally dependent


%%%%%%%%%%%%%%%%%%%%%%%%%%%%%%%%%%%%%%%%%%%%%%%%%%%%%%%%%%%%

% Alternative: put content in separate files
% Check the difference between including these files using \input{filename} and \include{filename} and see which one you like better
%\chapter{Einleitung}\label{intro}
%\input{introduction}
%
%\chapter{Voraussetzungen}\label{bg}
%\input{background}



\end{document}
