\documentclass[aspectratio=169,urlcolor=black]{beamer}
\usetheme[language=ngerman,
titlepagelogo=logopolito,
bullet=circle,
pageofpages=of,
color=blue,
titleline=true
]{TorinoTh}

%\usepackage[ngerman]{babel}
%\usepackage[utf8]{inputenc}
\usepackage{tabularx}
\usepackage{booktabs}
\usepackage{multicol}
\usepackage{ulem}
\usepackage{makecell}
\usepackage{upgreek}
\usepackage{movie15}
%\usepackage{xcolor}
%\usepackage{isodate}
\hypersetup{colorlinks,linkcolor=black,urlcolor=black}
\addto{\captionsngerman}{%
  \renewcommand*{\contentsname}{Contents}
  \renewcommand*{\listfigurename}{Figures}
  \renewcommand*{\listtablename}{Tables}
  \renewcommand*{\figurename}{Fig.}	
  \renewcommand*{\tablename}{Tab.}
}
\newcommand*\mean[1]{\bar{#1}}
\newcommand{\tabitem}{~~\llap{\textbullet}~~}
\newcommand\widebar[1]{\mathop{\overline{#1}}}
\usepackage{caption}
\captionsetup{font=scriptsize}

\usepackage{color}
\usepackage{graphicx}
\usepackage{fancybox}
\usepackage[singlespacing]{setspace}

\usepackage{beamerthemesplit}
\usetheme[compress]{Heidelberg}
\definecolor{unirot}{rgb}{0.4,0.4,0.3} % babyblue 0,0.58,1
\usecolortheme[named=unirot]{structure}
%\setbeamercolor{alerted text}{fg=red}
\newcommand*\hilite[1]{\textcolor{red}{#1}}
%\def\hilite<#1>{%
  %\temporal<#1>{\color{black}}{\color{unirot}}%
               %{\color{gray}}}


\title[Light Transport Techniques for Tensor Field Visualization]{Light Transport Techniques for Tensor Field Visualization}
\subtitle{Master's Thesis Presentation}
\vspace*{2pt}
\author[Sebastian Bek]{Sebastian Bek\vspace*{7pt}}
\date{July 24th 2019}
\institute[Uni HD]{
Heidelberg University\\
Visual Computing Group (VCG)\\
Supervisors: Prof. Filip Sadlo, Dr. Susanne Krömker\\
}

%\color{unirot}{sebibek@gmail.com}
%\newsubfloat{figure}
\newcommand{\source}[1]{\hspace{-3pt} {\tiny \raisebox{-0.75in}{\rotatebox[origin=t]{90}{Source: \color{black}{#1}}}}}
\setlength{\belowcaptionskip}{-10pt}
\setlength{\intextsep}{-20pt}
%\usepackage[style=verbose,backend=biber]{biblatex}
%\addbibresource{references.bib}
%\usepackage[multiple]{footmisc}
%\usepackage[bottom]{footmisc}
\renewcommand\thempfootnote{\arabic{mpfootnote}}
\begin{document}

%\frame[plain]{\titlepage}

\frame{
\frametitle{{Eigenanalysis - Tensor Fields}}
\begin{columns}
\begin{column}{.6\textwidth}
Direction, invariant under transformation $A$, sought:
\begin{block}{\centering \textbf{Eigenvalue Problem}}
\begin{align*}
	A \cdot \boldsymbol{\epsilon} = \lambda \boldsymbol{\epsilon}
\end{align*}
with: $A$: square matrix, $\lambda$: eigenvalue $\mathbf{\epsilon}$: eigenvector
\end{block}
\begin{itemize}
	\item eigensystems:\\ full set of eigenvalues and -vectors
	\item we propose to use singular value systems (SVS) in analogy to Eigensystems\\
	{\tiny Proof: "Glyphs for General Second-Order 2D and 3D Tensors", Gerrits et al., 2017}
\end{itemize}
\end{column}
\begin{column}{.4\textwidth}
\begin{figure}[t]
\begin{minipage}{0.85\textwidth}
\includegraphics[width=\textwidth]{HAT_edit2.png}
\caption*{Eigensystem: characteristic ellipsoid}
\end{minipage}
\begin{minipage}{0.1\textwidth}
\source{\url{https://slideplayer.com/slide/5291764}}
\end{minipage}

\end{figure}

\end{column}
\end{columns}
} % END OF FRAME

\frame{
\frametitle{{Computation of Eigensystem - Tensor Fields}}

\begin{columns}
\begin{column}{.6\textwidth}
\begin{small}
\begin{block}{\centering \textbf{Matrix Decompositions}}
1) Eigenvalue Decomposition: \hskip 46pt
$
	A = {R}{S}{R}^*
$\\
2) Singular Value Decomposition (SVD): 
$
	{A} = {U} {\Sigma} {V}^*
$
\end{block}
\end{small}
\begin{small}
\begin{enumerate}
	\item eigenvalues \hskip 14pt and eigenvectors
	\item singular values and singular vectors (SVs)
\end{enumerate}
\begin{itemize}
	\item SVs represent the axes of characteristic ellipsoid \footnote{Gerrits et al. ''Glyphs for General Second-Order 2D and 3D Tensors", IEEE Transactions on Visualization and Computer Graphics, 2017.}
	\item singular values ($s_i=\sqrt{\lvert\lambda_i\rvert}$)\footnote{M. Kieburg, “What is the Relation between Eigenvalues \& Singular
Values?, 2016} are lengths of axes
\end{itemize}
\end{small}

\end{column}
\begin{column}{.4\textwidth}
\begin{figure}[t]
\begin{minipage}{0.9\textwidth}
\includegraphics[width=\textwidth]{SVD.png}
\caption*{Singular value system}
\end{minipage}
\begin{minipage}{0.05\textwidth}
\source{\url{https://w.wiki/6WN}}
\end{minipage}
\end{figure}

\end{column}
\end{columns}
} % END OF FRAME


%glyphs are used to represent the eigensystem, i.e., the anisotropy characteristics and/or tensor magnitude
%Volume non-directionally dependent


%%%%%%%%%%%%%%%%%%%%%%%%%%%%%%%%%%%%%%%%%%%%%%%%%%%%%%%%%%%%

% Alternative: put content in separate files
% Check the difference between including these files using \input{filename} and \include{filename} and see which one you like better
%\chapter{Einleitung}\label{intro}
%\input{introduction}
%
%\chapter{Voraussetzungen}\label{bg}
%\input{background}



\end{document}
